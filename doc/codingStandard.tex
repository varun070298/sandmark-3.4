\section{Coding standard}
\begin{itemize}
    \item Don't use tab characters.
    \item Indent by typing three (3) blanks.
    \item Put the left brace ({\bf{\{}}) on
          same line as preceding statement. I.e.
          format an if-statement like this:
\begin{verbatim}
   if () {
      ...
   }
\end{verbatim}
        not like this:
\begin{verbatim}
   if ()
   {
     ...
   }
\end{verbatim}
        or like this:
\begin{verbatim}
   if ()
      {
        ...
      }
\end{verbatim}
     \item Don't use {\tt import}-statements. 
           Instead, always use fully qualified names.
           For example, say 
\begin{verbatim}
   java.util.Properties p = new java.util.Properties();
\end{verbatim}
           instead of
\begin{verbatim}
   import java.util.*;
   ...
   Properties p = new Properties();
\end{verbatim}
          In a large system like \SM\ it is difficult to
          read the code when you don't know where the
          a particular name is declared. In \SM\ we use
          deep package hierarchies to organize the code
          and always refer to every object using its 
          fully qualified name. This also prevents name
          clashes.
   \item We make one exception: you are allowed to say
         {\tt String} rather than {\tt java.lang.String}!
   \item We use the following naming strategy:
         \begin{itemize}
            \item Package names are short and in lower case.
                  We favor deep package hierarchies over
                  packages with many classes.
            \item Class names are typically long and descriptive.
                  They start with an uppercase letter.
            \item Method names start with a lowercase letter.
         \end{itemize}
   \item We make an effort to write methods that pass the ``hand test''.
	 The ``hand test'' passes if a method is short enough
	 to be completely covered by the programmer's hand placed horizontally
	 with fingers together on the screen.  (``Horizontally'' means that the
	 fingers are pointed to a side of the monitor, not toward the top or bottom.)
\end{itemize}

