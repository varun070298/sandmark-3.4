\documentclass{article}
\usepackage{url}
\usepackage{graphics}
\usepackage{fullpage}
\usepackage{epic}
\usepackage{eepic}
\usepackage{amsfonts}
\usepackage{fancyheadings}
\usepackage{moreverb}

\author{Richard Smith}
\title{A BLOAT Primer}

\begin{document}
\maketitle
\section{Introduction}

\subsection{What is BLOAT?}
      
BLOAT is a Java optimizer written entirely in Java. By
optimizing Java byte code, code improvements can occur
regardless of the compiler that compiled the byte code or the
virtual machine on which the byte code is run.  BLOAT performs
many traditional program optimizations such as constant/copy
propagation, constant folding and algebraic simplification,
dead code elimination, and peephole
optimizations. Additionally, it performs partial redundancy
elimination of arithmetic and field access paths.

Beyond this BLOAT can also be used for class creation as well
as method creation. In combination with byte code level
editing BLOAT is can also be used as a framework for Java
Class File obfuscation.
      
\subsection{What to expect?}

This how-to is in no way related to the creators of BLOAT. It
was written by Richard Smith, a computer science undergraduate
at the University of Arizona, to aid in research efforts by
Christian Collberg. This guide does not demonstrate the
optimization capabilities of BLOAT, but does demonstrate its
ability to be used as a framework for byte code
obfuscation. From this knowledge you can easily infer how to
do optimizations. For further questions about this how-to
email: \url{rsmith@cs.arionza.edu}.  You may also look at The BLOAT
Book which can be found at
\url{http://www.cs.purdue.edu/s3/projects/bloat}. At the time
of writing this how-to this book referred to BLOAT-0.8a which
lacks many of the features of BLOAT-1.0.

A number of examples of how BLOAT is used can be found
in the cvs directory \url{smbloat2}. 

\section{The Basics}

\subsection{Architecture}

 BLOAT is a very complicated beast. This will be a brief
overview, but will be enough to begin with. The first you
must understand is the concept of a context. A context
supplies a means of loading and editing classes. All contexts
implement the \url{EditorContext}. This interface supplies the
generic means of committing edits, editing classes, and a
utility to load classes. There are five major contexts:
\begin{enumerate}
  \item \url{BloatContext}, 
  \item \url{InlineContext}, 
  \item \url{EditorContext},
  \item \url{PersistentBloatContext}, and 
  \item \url{CachingBloatContext}. 
\end{enumerate}
The \url{InlineContext} and the \url{EditorContext} are both interfaces.  The
\url{InlineContext} and the \url{EditorClass} are both implemented by the
abstract class \url{BloatContext}. The \url{PersistentBloatContext} and
the \url{CachingBloatContext} are both subclasses of the
\url{BloatContext}. The two contexts that you should worry about are
the \url{PersistentBloatContext} and the \url{CachingBloatContext}. These
contexts are used for the same purpose, which is to act as a
repository for all of the edits made to java class files. The
difference between the two contexts is that the
\url{PersistentBloatContext} keeps every piece of data relevant to
your work, the \url{CachingBloatContext} on the other hand manages
your edits so that when a editor is no longer dirty it drops
the instance of the editor.


Editors consist of the \url{ClassEditor}, \url{MethodEditor}, and
\url{FieldEditor}. The names of these class should give evidence
enough of their subjects of manipulation.  It is through these
editors that you will make actual changes to byte code. It
should be noted that the \url{ClassEditor} constructor takes a
\url{BloatContext} and all other editors accepts a \url{ClassEditor}. In
some ways this is the use of a decorator pattern or wrapper
methodology.

\section{Editing Class Files}

\subsection{Where to Start?}

   BLOAT seems to have endless constructors that depend on other
objects which in turn depend on other objects. In truth unless
you know where to begin you will wrap yourself around the API
for a few hours trying to find the object you need create in
order to create the object you want to create.
   

The first thing that you need in BLOAT is a
\url{ClassFileLoader}. Unlike most classes in BLOAT the constructor
for this class takes no arguments. The \url{ClassFileLoader}
provides you with tools to manipulate the system CLASSPATH,
open zip/jar archives, and create new classes.  Overall these
tools will not be useful to you. They can be used as opposed
to the process I am about to demonstrate to you. However, if
you chose to do so you will not be using contexts as the
central point of your BLOAT manipulations which was fore
mentioned as the core of the BLOAT architecture.
   

After creating a \url{ClassFileLoader} the next step is to create
either a \url{PersistentBloatContext} or a \url{CachingBloatContext}. The
choice between these two contexts depends on the resources
available to you. In most case you should chose the
\url{PersistentBloatContext} as memory is cheap.


   Once you have a BLOAT context you can load the class files you
wish to edit. This is done by using the \url{loadClass} method of
the BLOAT context. You could also use the same method in the
\url{ClassFileLoader}, however again this does not take advantage of
the context architecture.


After loading the class file the \url{loadClass} method returns a
\url{ClassInfo} object. This is really an interface implemented by
the \url{ClassFile} object so you will want to cast the \url{ClassInfo}
object to a \url{ClassFile} to take advantage of some more advanced
features found in the \url{ClassFile} object. The \url{ClassFile} object,
as one might think, contains all of the relevant information
about the class you load. This includes information pertaining
to fields and methods as well as more byte code specific
information. Figure~\ref{BLOAT-fig1} will illustrate this process so far.

\begin{figure}
\begin{listing}{1}
EDU.purdue.cs.bloat.file.ClassFileLoader classFileLoader = new
   EDU.purdue.cs.bloat.file.ClassFileLoader();

EDU.purdue.cs.bloat.context.PersistentBloatContext bloatContext = 
   new EDU.purdue.cs.bloat.context.PersistentBloatContext(
       classFile.loader());

String className = "ClassFileToTest";
EDU.purdue.cs.bloat.reflect.ClassInfo info = null;

try { 
   info = bloatContext.loadClass(className); 
} catch (ClassNotFoundException ex) {
   System.err.println("Couldn't find class: " +ex.getMessage()); 
   System.exit(1); 
}

EDU.purdue.cs.bloat.file.ClassFile classFile =
      (EDU.purdue.cs.bloat.file.ClassFile)info;
\end{listing}
\caption{Loading a class file.}
\label{BLOAT-fig1}
\end{figure}

\subsection{The \protect\url{ClassEditor}}
   
Once you created a \url{ClassFile} object you can begin byte-code
manipulation. The next step is to create a \url{ClassEditor}. There
are two ways to go about this, create a \url{ClassEditor} from
scratch or create a \url{ClassEditor} from an existing \url{ClassFile}.
   

To create a \url{ClassEditor} from an existing \url{ClassFile} simply use
the method \url{editClass} supplied by the \url{BloatContext}. This method
returns a \url{ClassEditor} for the \url{ClassFile} passed as a
parameter:
\begin{listing}{1}
EDU.purdue.cs.bloat.editor.ClassEditor classEditor =
              bloatContext.editClass(classFile);
\end{listing}

To create a \url{ClassEditor} from scratch is a little
different. BLOAT supplies a constructor for the \url{ClassEditor} to
do so however the API advises you not to use this.  Instead
you should use the method \url{newClass} provided by the
\url{BloatContext}:
\begin{verbatim}
newClass(
   int�modifiers, 
   java.lang.String�className,
   Type�superType, 
   Type[]�interfaces);
\end{verbatim}

The \url{newClass} method takes a modifier, a String, a Type, and a
Type array. The modifier specifies whether the class will be
public, private, or protected. The correct parameter to use
for the modifier is a constant from the modifier interface
found in the \url{EDU.purdue.cs.bloat.reflect} package:
\begin{verbatim}
EDU.purdue.cs.bloat.reflect.Modifier.PUBLIC
EDU.purdue.cs.bloat.reflect.Modifier.PRIVATE
EDU.purdue.cs.bloat.reflect.Modifier.PROTECTED
\end{verbatim}

The String parameter specifies the name of the class being
created. The Type passed to the constructor indicates from
where the class extends i.e. java.lang.Object. Again this is
specified by a constant from the \url{EDU.purdue.cs.bloat.editor}
package:
\begin{verbatim}
EDU.purdue.cs.bloat.editor.Type.OBJECT
\end{verbatim}

The Type array specifies all of the interfaces that the class
is to implement. Again these can be found in the
\url{EDU.purdue.cs.bloat.editor} package and include interface such
as Serializable and Comparable. Here's a complete
example of the use of this method:
\begin{listing}{1}
EDU.purdue.cs.bloat.editor.ClassEditor classEditor =
   bloatContext.newClass(
      EDU.purdue.cs.bloat.reflect.Modifiers.SUPER,
      "MyNewClass",
      EDU.purdue.cs.bloat.editor.Type.OBJECT, 
      null);
\end{listing}

Note that if you intend to implement an interface or override
a method that is inherited you will have to create those
methods from scratch as well. The section on the method editor
will demonstrate method creation from scratch.


The \url{ClassEditor} provides manipulation of data such as the
class modifiers, the constant pool, and the interfaces
implemented by the class. All of these uses are pretty
intuitive with the exception of constant pool
manipulation. You should refer to \url{ChangeFieldNames.java},
provided in the directory examples-1.0, for an example of
constant pool manipulation.

\subsection{The \protect\url{MethodEditor}}
   
If you wish to edit a method in an existing class you will
again need to refer to the \url{BloatContext}. The \url{BloatContext}
provides a method, \url{editMethod}, to modify methods. The
\url{editMethod} method takes a \url{MethodInfo} object as a
parameter. You can retrieve this information from a
\url{ClassEditor} with the method,\url{methods}, that returns an array of
\url{MethodInfo} objects:
\begin{listing}{1}
EDU.purdue.cs.bloat.reflect.MethodInfo[] methods =
   classEditor.methods();
EDU.purdue.cs.editor.MethodEditor methodEditor =
   bloatContext.editMethod(methods[0]);
\end{listing}
This array contains the method information
for all of the methods contained in the \url{ClassFile} related to
the class editor including the constructor. Note that a
\url{MethodInfo} object is really an interfaced used by a \url{Method}
object and therefore you can cast a \url{MethodInfo} object to a
\url{Method} object.

   There is most likely a specific method that you wish to
edit. You can retrieve a specific method by obtaining its name
with the name method in the \url{MethodEditor} object. To create a
new method for an existing class or a class you creating from
scratch you must create a new \url{MethodEditor} object.  Note that
unlike creating a \url{ClassEditor} from scratch this is not done
through the BLOAT context.  To create a new \url{MethodEditor} use
the constructor provided below:
\begin{verbatim}
MethodEditor(
   ClassEditor�editor, 
   int�modifiers,
   Type�returnType, 
   java.lang.String�methodName,
   Type[]�paramTypes, 
   Type[]�exceptionTypes);
\end{verbatim}

The constructor first takes a \url{ClassEditor} for the class you
wish to create a new method in.  Then you must provide a
modifier such as public, private, or protected. This can be
specified by a constant in the \url{EDU.purdue.cs.reflect.Modifier}
interface. The next parameter specifies the return type of the
method. This can be specified using a constant from the
\url{EDU.purdue.cs.editor.Type} interface. The String parameter is
used to name the method you have chosen to create. Then the
parameters of the method are specified by constants from the
Type interface. The final parameter of the constructor
specifies exceptions thrown by the method. Again this can be
specified using constants found in the Type interface.

Once you have created a method you can begin adding
instructions to the method. This done by calling the method
\url{addInstruction} in the \url{MethodEditor}. The method takes an opcode
constant as a parameter. These constants can be found in the
\url{EDU.purdue.cs.bloat.editor.Opcode} interface.  There is an
opcode in this interface corresponding to each opcode in Java
byte-code.  It should be noted that valid Java byte-code must
be entered or you will receive a Java runtime error.

\subsection{The \protect\url{FieldEditor}}
   
The \url{FieldEditor} is used to edit attributes of a class. To edit
fields of an existing class simply call the \url{editField} method
of the \url{BloatContext}. This method takes a \url{FieldInfo}
object. This object can be obtained from the \url{ClassEditor} by
calling the fields method. This method returns an array of
\url{FieldInfo} objects. This example demonstrates how to put these
things together to create a \url{FieldEditor}:
\begin{listing}{1}
EDU.purdue.cs.bloat.editor.FieldInfo[] fieldInfo =
   classEditor.fields();
EDU.purdue.cs.bloat.editor.FieldEditor fieldEditor = 
   bloatContext.editField(fieldInfo[0]);
\end{listing}

To create a new field you must use a constructor found in the
\url{FieldEditor} class.  The constructor is illustrated below:
\begin{verbatim}
FieldEditor(
   ClassEditor�editor, 
   int�modifiers,
   java.lang.Class�type, 
   java.lang.String�name);
\end{verbatim}


The constructor takes the \url{ClassEditor} for the class you wish to
add an attribute to. Like the constructor to create a new
method the constructor for the \url{FieldEditor} also takes a
constant to specify a Modifier to specify whether the
attribute is public, private, or protected. The constructor
also takes a Type constant to indicate whether the attribute
is an int, String, or other possibility. Finally the
constructor accepts a String to name the attribute.

\section{Committing Edits}

\subsection{One Rule To Follow}
   
There is one rule to follow in BLOAT that will save you a lot
of pain, after you edit commit.  This seems pretty obvious but
there are al lot of methods to \url{commit} edits in BLOAT. For
instance the \url{ClassEditor}, \url{FieldEditor}, \url{MethodEditor}, and the
\url{BloatContext} all have several commit methods. The rule can be
applied as followed. If you make an edit with a \url{FieldEditor},
or any other editor, call the \url{commit} method of the editor when
you have completed editing with that specific editor. The
\url{BloatContext}'s  \url{commit} method can be used to commit all
edits to all class files the \url{BloatContext} has knowledge of,
i.e. any class that you have loaded or created through the
\url{BloatContext}.


\section{Instruction Representation}

\subsection{Opcodes}
The folks who created BLOAT were nice enough to create a constant to
represent every opcode in the Java byte-code language. These opcodes
can be found in the \url{EDU.purdue.cs.bloat.editor.Opcode} interface.

\subsection{Instructions}
You can create an instruction by simply passing an Instruction object
an opcode constant form the \url{EDU.purdue.cs.bloat.editor.Opcode}
interface.

\begin{listing}{1}
EDU.purdue.cs.bloat.editor.Instruction instruction = 
   new EDU.purdue.cs.bloat.editor.Instruction(
       EDU.purdue.cs.bloat.editor.Opcode.opc_astore);
\end{listing}

There are several methods in the Instruction class that test whether
the instruction the class represents is a jump or a switch as well as
many other things. Each of these methods returns a boolean value based
on their validity.

Each instruction can be add to a method or block in a control flow
graph. You can do this by calling the \url{addInstruction} method for either
of these objects and passing it an Instruction object.

\section{Advanced Features}
\subsection{Inlining}
Inlining is a fairly straight forward process. Simply create a
\url{BloatContext} and load your classes as usual. You will now need to
create an \url{Inline} object. The constructor for the object takes a
\url{InlineContext} and an integer as a parameter. There are no special
steps for obtaining an \url{InlineContext} the \url{BloatContext} already
implements this interface. The integer determines the maximum number
of instructions a method can grow to.

\begin{listing}{1}
EDU.purdue.cs.bloat.inline.Inline inline = 
   new EDU.purdue.cs.bloat.inline.Inline(bloatContext,20);
\end{listing}

To actually inline a method you must call the \url{inline} method of the
Inline class. This method takes a \url{MethodEditor} as a parameter. Refer
to the section on \url{MethodEditor}s if you do not already know how to
create one. Once the method is called the inlining is complete:

\begin{listing}{1}
EDU.purdue.cs.bloat.inline.Inline inline = 
   new EDU.purdue.cs.bloat.inline.Inline(bloatContext,20);
inline.inline(methodEditor);
\end{listing}

 
There are some extra features that can be used during inlining. Using
the method \url{maxCallDepth} you can specify with an integer the maximum
number of nested method calls inlined. The inlining process can be
rather time consuming. Bloat attempts to remedy this by providing a
method, \url{setMaxInlineSize}, that sets the maximum size method that will
be inlined. This size is set by passing an integer to the
\url{setMaxInlineSize} method.You may also specify whether to inline methods
that throw exceptions or not. This is simply set passing a boolean
value to the method \url{setInlineExceptions}. Note that if you need to use
one these features you must call one of these methods prior to calling
the \url{inline} method.

\begin{listing}{1}
int size = 25; 
int depth = 3; 
boolean flag = true;
inline.setMaxInlineSize(size); 
inline.setMaxCallDepth(depth);
inline.setInlineExceptions(flag); 
inline.inline(methodEditor);
\end{listing}


\subsection{Class Hierarchy}
It maybe of some interest to understand or manipulate the class
hierarchy of a class. The first step in this process is to load all
classes in the hierarchy using your current \url{BloatContext}. The
\url{BloatContext} must first have knowledge of all of the class relevant to
the hierarchy. Once the classes are load you can extract information
on the hierarchy by calling \url{getHierarchy} method found in the
\url{BloatContext}. The method returns a \url{ClassHierarchy} object. The
\url{ClassHierarchy} object allows you to view all of the classes and
interfaces in a hierarchy as well as edit the hierarchy by adding or
removing classes an interfaces.

\begin{listing}{1}
// Add a class 
EDU.purdue.cs.bloat.editor.ClassHierarchy classHierarchy =
   bloatContext.getHeirarchy(); 
String className = "Test"; 
classHierarchy.addClassNamed(Test);  
// extract the Type of classes in the hierarchy:
Collection classes = 
   classHierarchy.classes(); 
// extract Type of classes that implement a specific interface:
Collection classes = 
   classHierarchy.implementors(
      EDU.purdue.cs.bloat.editor.Type.SERIALIZABLE);
// find interfaces a Type of object implements 
Collection interfaces = 
   classHierarchy.interfaces(
      EDU.purdue.cs.bloat.editor.Type.CLASS); 
// extract subclasses of a Type of object 
Collection subclasses = 
   classHierarchy.subclasses(EDU.purdue.cs.bloat.editor.Type.OBJECT);
\end{listing}


The \url{ClassHierarchy} may also be used to determine what method will be
invoked at any depth in the hierarchy. Therefore you can determine
from where a method is inherited or overwritten.

\begin{verbatim}
// simulates dynamic dispatching: 
MemberRef methodInvoked(
   Type receiver, 
   NameAndType method);
\end{verbatim}

The method takes two arguments a Type, which we are already familiar
with from previous examples, and a \url{NameAndType}. The Type argument
determines the receiver of the call. The \url{NameAndType} class simply
describes an a method by its name and descriptor. The name is simply a
String and the descriptor describes the return type of the method
using the Type class.

\begin{verbatim}
EDU.purdue.cs.bloat.editor.NameAndType(
   java.lang.String name, 
   Type type);
\end{verbatim}

The \url{MemberRef} returned by \url{methodInvoked} represents the method you are
interested in and class from where it was invoked.

\begin{listing}{1}
//Putting it all together:
EDU.purdue.cs.bloat.editor.ClassHierarchy classHierarchy = 
   bloatContext.getHeirarchy();
EDU.purdue.cs.bloat.editor.NameAndType nameAndType = 
   new EDU.purdue.cs.bloat.editor.NameAndType(
      "test",
      EDU.purdue.cs.bloat.editor.Type.STRING);
MemberRef memRef =
   classHeirarchy.methodInvoked(
      EDU.purdue.cs.bloat.editor.Type.OBJECT,
      nameAndType);
System.out.println(memRef);
\end{listing}

Using the \url{ClassHierarchy} you may also determine if a method has been
overwritten. Note the previous example could be used to determine
where it was overwritten. In order to determine if a method has been
overwritten you must call the \url{methodIsOverridden} method:

\begin{verbatim}
boolean methodIsOverridden(
   Type classType, 
   NameAndType method);
\end{verbatim}

To use this method you must pass it the class Type of where the method
resides and a \url{NameAndType} that identifies the method and its return
type. The method will return a boolean value based on whether the
method has been overwritten for that class Type.

\begin{listing}{1}
//Putting it all together 
EDU.purdue.cs.bloat.editor.ClassHierarchy classHierarchy = 
   bloatContext.getHeirarchy();
EDU.purdue.cs.bloat.editor.NameAndType nameAndType = 
   new EDU.purdue.cs.bloat.editor.NameAndType(
      "test",
      EDU.purdue.cs.bloat.editor.Type.STRING);
boolean value =
   classHeirarchy.methodIsoverridden(
      EDU.purdue.cs.bloat.editor.Type.OBJECT,
      nameAndType);
\end{listing}

\subsection{Control Flow Graph (CFG)}
Bloat allows a developer to create a control flow graph for any given
method. To do so you simply need to construct a \url{FlowGraph} object. To
create a \url{FlowGraph} you must provide a \url{MethodEditor} for the constructor
the \url{FlowGraph}.

\begin{verbatim}
EDU.purdue.cs.bloat.cfg.FlowGraph(
   MethodEditor method);
\end{verbatim}

To construct the \url{FlowGraph} object does not setup the cfg. You must
call the initialize method of the \url{FlowGraph} object to create the
graph. Once the graph is created you can extract the basic blocks
using a variety of methods. Most of these methods produce a \url{List} or
\url{Collection} of blocks that can be iterated across. There also several
\url{print} methods to print the CFG in various orderings.

\begin{listing}{1}
// printing cfg in different orders 
EDU.purdue.cs.bloat.cfg.FlowGraph cfg = 
   new EDU.purdue.cs.bloat.cfg.FlowGraph(methodEditor);
cfg.preOrder(); 
cfg.postOrder(); 
cfg.print(); 
cfg.printGraph();
\end{listing}


\begin{listing}{1}
// extracting blocks and manipulating blocks:
EDU.purdue.cs.bloat.cfg.FlowGraph cfg = 
   new EDU.purdue.cs.bloat.cfg.FlowGraph(methodEditor);
//Blocks returned in trace order:
java.util.List list =
   cfg.trace(); 
for(int i=0;i<list.size();i++){ 
   EDU.purdue.cs.bloat.tree.Tree tree =
      list.tree(); 
   EDU.purdue.cs.bloat.editor.Instruction instruction = 
      new EDU.purdue.cs.bloat.editor.Instruction(
         EDU.purdue.cs.bloat.editor.Opcode.ops_aaload);
   tree.addInstruction(instruction); 
}
\end{listing}
 
The above example extracts the blocks from a cfg. It then accesses the
Tree for each Block. A Tree represents the expression tree for each
block. Using the tree you can view the instructions contained with in
a Block as well as edit them by adding and deleting instructions. The
example above demonstrates how to add an instruction.

\subsection{Trees}
Trees were previously introduced in the section on \url{FlowGraph}. \url{Tree} 
represents the expression tree of a basic block. It consists of an operand 
(expression) stack comprised of expressions and a list of statements contained 
in the block. The blocks, as previously mentioned, can be obtained from the 
\url{FlowGraph}. \url{Block} contains a method called \url{tree} that takes no arguments
and returns a \url{Tree}. \url{Tree} allows you to manipulate the expression 
tree for the from which obtain \url{Tree}. Instructions maybe added with methods
below.

\begin{listing}{1}
// adds an instruction that does not change control flow
addInstruction(Instruction inst) 
// adds an instruction to jump to another Block in the FlowGraph
addInstruction(Instruction inst, Block next)
// add an instruction such as ret or astore to invoke a subroutine
addInstruction(Instruction inst, Subroutine sub) 
\end{listing}
 
A branch in the cfg maybe labeled using that {\tt addLabel(Label method)}. To create a 
label follow the example below. The index used in the example represents the labels
in the instruction array. The boolean in the constructor specifies wheither the 
\url{Label} start a block or not. You may also add a comment to a \url{Label} by calling 
the \url{setComment} method which takes a string to represent the comment.

\begin{listing}{2}
// The index represents represents a unique offset, .i.e. its offset
// in the instruction array
int index = 23;
EDU.purdue.cs.bloat.editor.Label label = new EDU.purdue.cs.bloat.editor.Label(index,false);
label.setComment(``Hello'');
System.out.println(label);
tree.addLabel(label);
\end{listing}

An additional \url{Stmt} can also be added to a \url{Tree} using the method {\tt addStmt(Stmt stmt)}. 
A \url{Stmt} is the super class of \url{JumpStmt}, \url{ExprStmt}, and \url{PhisStmt}. A \url{JmpStmt} 
is a super class of the following: \url{GotoStmt},\url{IfStmt},\url{JsrStmt}. A \url{GotoStmt} represents an unconditional 
branch to a basic block. The constructor takes a \url{Block} as an argument which represents the 
target of the \url{GotoStmt}. An \url{IfStmt} is a super class of statements in which some expression 
is evaluated and one of two branches is taken. Its subclasses are \url{IfCmpStmt} and \url{IfZeroStmt}.
To create an \url{IfCmpStmt} the constructor takes 5 arguments. The first is an integer that represents
the comparison operator the if statement. Next it takes a left \url{Expr} and right \url{Expr}. Finally
it takes a \url{Block} to jump to when the if condition is true and then a \url{Block} to jump to if the 
condition is false. An \url{IfZeroStmt} evaluates an expression and executes one of its two branches 
depending on whether or not the expression evaluated to zero. To create a \url{IfZeroStmt} you simply
use the same arguement sequence as supplied to a \url{IfCmpStmt}. You should take note that a \url{Stmt} 
can not be nested as \url{Expr} can. These are not all of the \url{Stmt} supplied by BLOAT. This is only 
an introduction into what they offer. \url{Stmt}s can be removed from a \url{Tree}
using the method {\tt removeStmt(Stmt stmt)}.


\subsection{SSA (Value Graphs)}
A \url{FlowGraph} object may also be used to create a value graph. The SSA
graph (also called the value graph) represents the nesting of
expression in a control flow graph. Each node in the SSA graph
represents an expression.  If the expression is a definition, the it
is labeled with the variable it defines. Each node has directed edges
to the nodes representing its operands.

\url{SSAGraph} is a representation of the definitions found in a CFG in the
following form: Each node in the graph is an expression that defines a
variable (a \url{VarExpr}, \url{PhiStmt}, or a \url{StackManipStmt}). 
Edges in the graph point to the nodes whose expressions define the 
operands of the expression in the source node.

This class is used primarily get the strongly connected components of
the SSA graph in support of value numbering and induction variable
analysis.

Nate (One of the Authors of BLOAT) warns: Do not modify the CFG while
using the SSA graph! The effects of such modification are undefined
and will probably lead to nasty things occurring.

\begin{listing}{1}
EDU.purdue.cs.bloat.ssa.SSAGraph ssaGraph = 
   new EDU.purdue.cs.bloat.ssa.SSAGraph(cfg);
\end{listing}

\subsection{Type-Based Alias Analysis (TBBA)}

Type-Based Alias Analysis (TBAA) is used to determine if one expression 
can alias another. An expression may alias another expression if there is 
the possibility that they refer to the same location in memory. BLOAT 
models expressions that may reference memory locations with \url{MemRefExprs}. 
There are two kinds of "access expressions" in Java: field accesses 
(\url{FieldExpr}  and \url{StaticFieldExpr}) and array references (\url{ArrayRefExpr}). Access 
paths consist of one or more access expressions. For instance, {\tt a.b[i].c} is an 
access expression.

TBAA uses the \url{FieldTypeDecl} relation to determine whether or not two access 
paths may refer to the same memory location.

A static method in the \url{EDU.purdue.cs.bloat.tbaa.TBAA} class may be used to do
Type-Based Alias Analysis.

\begin{listing}{1}
 EDU.purdue.cs.bloat.tbaa.TBAA.canAlias(EditorContext context, Expr a, Expr b) 
\end{listing}

Where the \url{EditorContext} represents the \url{BloatContext} and the \url{Expr} parameters represent
expressions. \url{EDU.purdue.cs.bloat.tree.Expr} is a an abstract superclass. Its known subclasses
are listed below.

\begin{listing}{1}
ArithExpr
ArrayLengthExpr
CallExpr
CastExpr
CatchExpr
CheckExpr
CondExpr
ConstantExpr
DefExpr
NegExpr
NewArrayExpr
NewExpr
NewMultiArrayExpr
ReturnAddressExpr
ShiftExpr
StoreExpr
\end{listing}


\end{document}