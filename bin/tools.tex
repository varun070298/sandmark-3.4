\section{Examining Java Classfiles}
There are a number of tools that are helpful for
viewing Java classfiles.

\subsection{javap}
{\tt javap} lists the contents of a Java classfile. 
It's particularly bad at displaying corrupted 
classfiles.

Normally, we call {\tt javap} like this:
\begin{verbatim}
   javap -c -s -verbose -l TTTApplication
\end{verbatim}

These are the available options:
\begin{verbatim}
Usage: javap <options> <classes>...

where options include:
   -b                        Backward compatibility with javap in JDK 1.1
   -c                        Disassemble the code
   -classpath <pathlist>     Specify where to find user class files
   -extdirs <dirs>           Override location of installed extensions
   -help                     Print this usage message
   -J<flag>                  Pass <flag> directly to the runtime system
   -l                        Print line number and local variable tables
   -public                   Show only public classes and members
   -protected                Show protected/public classes and members
   -package                  Show package/protected/public classes
                             and members (default)
   -private                  Show all classes and members
   -s                        Print internal type signatures
   -bootclasspath <pathlist> Override location of class files loaded
                             by the bootstrap class loader
   -verbose                  Print stack size, number of locals and args for methods
                             If verifying, print reasons for failure
\end{verbatim}

\subsection{Jasmin}
{\tt Jasmin} is a Java bytecode assembler. It reads
a text file containing Java bytecode instructions and
generates a classfile. It can be found here:
\url{http://mrl.nyu.edu/~meyer/jasmin/about.html}.

Here's is a simple class {\tt hello.j} in the {\tt Jasmin}
assembler syntax:
\begin{listing}{1}
    .class public HelloWorld
    .super java/lang/Object

    ;
    ; standard initializer (calls java.lang.Object's initializer)
    ;
    .method public <init>()V
       aload_0
       invokenonvirtual java/lang/Object/<init>()V
       return
    .end method

    ;
    ; main() - prints out Hello World
    ;
    .method public static main([Ljava/lang/String;)V
       .limit stack 2   ; up to two items can be pushed

       ; push System.out onto the stack
       getstatic java/lang/System/out Ljava/io/PrintStream;

       ; push a string onto the stack
       ldc "Hello World!"

       ; call the PrintStream.println() method.
       invokevirtual java/io/PrintStream/println(Ljava/lang/String;)V

       ; done
       return
    .end method
\end{listing}

Call {\tt Jasmin} like this:
\begin{verbatim}
   java -classpath smextern/jasmin.jar jasmin.Main hello.j
\end{verbatim}
which will produce a class file {\tt HelloWorld.class}.


\subsection{BCEL's Class Construction Kit}
{\tt cck} is an interactive viewer and editor of Java classfiles.
It is built on {\tt BCEL}. Call it like this:
\begin{verbatim}
   java -classpath .:smextern/BCEL.jar -jar smextern/cck.jar
\end{verbatim}
and open a classfile from the {\tt file}-menu.


\subsection{BCEL's listclass}
{\tt listclass} comes with the BCEL package. It's
a replacement for {\tt javap}, and very useful
in cases when {\tt javap} crashes. Call it like
this:
\begin{verbatim}
   java -classpath smextern/BCEL.jar listclass -code TTTApplication
\end{verbatim}

The following options are available:
\begin{verbatim}
    java listclass [-constants] [-code] [-brief] [-dependencies] \
         [-nocontents] [-recurse] class... [-exclude <list>]
\end{verbatim}

\begin{description}
   \item[{\tt -code}:] 
       List byte code of methods.
   \item[{\tt -brief}:] 
       List byte codes briefly
   \item[{\tt -constants}:] 
       Print constants table (constant pool)
   \item[{\tt -recurse}:]  
       Usually intended to be used along with
   \item[{\tt -dependencies:}]
       When this flag is set, listclass will also print i
\end{description}

\subsection{BCEL's JustIce}
{\tt JustIce} verifies classfiles. Call it like this:
\begin{verbatim}
    java -classpath .:smextern/BCEL.jar:smextern/JustIce.jar \
      org.apache.bcel.verifier.GraphicalVerifier
\end{verbatim}
or like this:
\begin{verbatim}
   java -classpath .:smextern/BCEL.jar:smextern/JustIce.jar \
      org.apache.bcel.verifier.Verifier TTTApplication.class
\end{verbatim}
