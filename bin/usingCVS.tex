\section{Using CVS}
CVS is a {\em source code control system}. It is used extensively
in industry and in the open source community to organize source
code that many people are working on simultaneously. 

The basic idea is to have one master copy of the source code,
residing in a special source code repository.
Programmers check out the latest version of the code to
their local account and work on it until they're ready to
share their new code with their co-workers. They then
check in the new code to the repository from which the
rest of the team can download the new changes. 

A programmer can have several versions of the code checked 
out at the same time: one at work, one at home, one on the
laptop, etc. Furthermore, every change ever made to a file
is stored in the repository allowing a programmer to check
out ``the version of the program from last Monday.''


\subsection{Installing CVS}
If you are running Linux on your home machine (and why wouldn't
you be?) you should have CVS installed already. Otherwise you
can download CVS from here: \url{http://www.cvshome.org/downloads.html}.
The manual is here: \url{http://www.cvshome.org/docs/manual}. 
\CC{man cvs} gives you the basic information you need.

You also need to have {\tt ssh} installed on your machine
in order to communicate with our CVS server, 
\url{cvs.cs.arizona.edu}.

If you just intend to do your assignments on {\tt lectura},
no installation is necessary.

\subsection{Getting Started}
Let's assume that your team consists of Alice and Bob whose
{\tt lectura} logins are {\tt alice} and {\tt bob} with
the passwords {\tt alice-pw} and {\tt bob-pw}, respectively.
The team name is {\tt cs453bd}.

{\em One} team member (in our case Alice) should do the following 
to get the team's source repository set up:
\begin{verbatim}
> whoami
alice
> mkdir ass1            # Create a temorary directory.
> cd ass1
> setenv CVS_RSH ssh    # Or export CVS_RSH=ssh. Must always be set.
> cvs -d :ext:alice@cvs.cs.arizona.edu:/cvs/cvs/cs453/cs453bd import -m "New!" ass1 aaa bbb ass1
   alice@cvs.cs.arizona.edu's password: alice-pw
\end{verbatim}

Now, Alice deletes the temporary directory:
\begin{verbatim}
> rmdir ass1
\end{verbatim}


\subsection{Checking out code}
Everything should now be set up properly on the CVS server.
Alice can check out the code (which so-far only consists of a
single directory):
\begin{verbatim}
> cvs -d :ext:alice@cvs.cs.arizona.edu:/cvs/cvs/cs453/cs453bd checkout ass1
   alice@cvs.cs.arizona.edu's password: alice-pw
cvs server: Updating ass1
> ls ass1
CVS
\end{verbatim}

Alice now wants to start programming. She creates a new C module
in her CVS directory:
\begin{verbatim}
> cd ass1
/home/alice/ass1
> cat > interpreter.c
main() { 
}
> cvs add -m "Started the project" interpreter.c 
   alice@cvs.cs.arizona.edu's password: alice-pw
   cvs server: scheduling file `interpreter.c' for addition
   cvs server: use 'cvs commit' to add this file permanently

> cvs commit -m "Finished first part of interpreter."
   cvs commit: Examining .
   alice@cvs.cs.arizona.edu's password: 
   RCS file: /cvs/cvs/cs453/cs453bd/ass1/interpreter.c,v
   done
   Checking in interpreter.c;
   /cvs/cvs/cs453/cs453bd/ass1/interpreter.c,v  <--  interpreter.c
   initial revision: 1.1
   done
\end{verbatim}

The {\tt add} command told the CVS system that a new file 
is being created. The {\tt commit} command actually uploaded
the new file to the repository.

Now Alice realizes that she needs to add some more code
the project:
\begin{verbatim}
> emacs interpreter.c 
> cat interpreter.c 
main() { 
  int i;
}
> cvs commit -m "Added more code."
   cvs commit: Examining .
   alice@cvs.cs.arizona.edu's password: alice-pw
   Checking in interpreter.c;
   /cvs/cvs/cs453/cs453bd/ass1/interpreter.c,v  <--  interpreter.c
   new revision: 1.2; previous revision: 1.1
   done
\end{verbatim}

OK, so what about Bob? Well, he decides he should also contribute
to the project, so he checks out the source:
\begin{verbatim}
> cvs -d :ext:bob@cvs.cs.arizona.edu:/cvs/cvs/cs453/cs453bd checkout ass1
   bob@cvs.cs.arizona.edu's password: bob-pw
cvs server: Updating ass1
> cd ass1
> ls
CVS interpreter.c
> cat interpreter.c 
main() { 
  int i;
}
> emacs interpreter.c 
> cat interpreter.c 
main() { 
  int i=5;
}
> cvs commit -m "Added more stuff to the project."
\end{verbatim}

Alice has now gone back to her dorm-room where she wants to
continue working on the project on her home computer. She
has installed CVS and she has added
\begin{verbatim}
> setenv CVS_RSH ssh    # Or export CVS_RSH=ssh
\end{verbatim}
to her {\tt .cshrc} file to make sure that she runs
this command every time. Now she can go ahead and
check out the code again, this time on the home 
machine: 
\begin{verbatim}
> cvs -d :ext:alice@cvs.cs.arizona.edu:/cvs/cvs/cs453/cs453bd checkout ass1
   alice@cvs.cs.arizona.edu's password: alice-pw
> cat ass1/interpreter.c 
main() { 
  int i=5;
}
\end{verbatim}
Notice that she got the code that Bob checked in to CVS!

Alice can continue working on the code from home. When
she's done for the day she uses the {\tt commit} command
to submit her changes to the cvs database. 

\subsection{Updating}
The next day Bob is getting ready to work on the project again.
In case Alice has made some changes to the code, he runs
the {\tt update} command:
\begin{verbatim}
> cvs update -d
\end{verbatim}
Any files that have changed since the last time Bob worked
on the project will be downloaded from the server. Bob
makes his edits, then runs {\tt commit} when he is done to
upload the changes to the repository.

\subsection{Deleting files}
If Alice needs to delete a file she runs the CVS {\tt rm} command:
\begin{verbatim}
> rm interpreter.c
> cvs rm interpreter.c
> cvs commit
\end{verbatim}
Note that you have to delete the file before you can run the
{\tt cvs rm} command.

\subsection{Summary}
These are the most common CVS commands:
\begin{description}
   \item[cvs add file] Add a new file to the project.
         The file will not actually be uploaded to the
         repository until you run the {\tt commit} command.
   \item[cvs rm file] Remove a file from the project.
         The file will not actually be removed from the
         repository until you run the {\tt commit} command.
   \item[cvs commit] Update the repository with any changed
         files.
   \item[cvs update -d] Download any changed files to your
        local machine.
\end{description}

The figure below describes a typical situation.
Alice and Bob have three versions of the code checked out:
two on their lectura accounts, and one version in Alice's
home machine. Alice adds a new file {\tt file3.c} and 
checks it in to the repository. To see the new file, 
Bob has to run the {\tt update} command.
\begin{center}
   If someone has added a related comment please add your comment in the same area.
\begin{enumerate}
\item Add your lessons here. (Please erase once a lesson has been added)
\end{enumerate}

\end{center}
