\section{BCEL}
\SM\ reads Java classfiles, modifies them, and
writes out the modified classfiles. There are
a number of free packages available to parse,
modify, and unparse classfiles. We are currently
using two such packages, namely {\tt BCEL} and
{\tt BLOAT}. Generally, {\tt BLOAT} is comprehensive
and hard to use, {\tt BCEL} is simpler to use but
not as complete.

\BCEL\ (formerly {\tt JavaClass}) allows you to load a class,
iterate through the methods and fields, change
methods, add new methods and fields, etc. \BCEL\
also makes it easy to create new classes from
scratch.

\BCEL\ is here:
\url{http://jakarta.apache.org/bcel/index.html}.
There is a manual, and an on-line API description
at \url{http://bcel.sourceforge.net/docs/index.html}.
Unfortunately, neither are particularly informative.

\SM\ contains a lot of \BCEL\ code that might be
worth studying:
\begin{itemize}
   \item \url{sandmark.util.javagen} is a package
         for building up bytecode classes from
         scratch. Essentially, it allows you to
         build a Java AST (abstract syntax tree)
         and compile it down to bytecode. \url{javagen}
         is built on top of \BCEL.
   \item \url{sandmark.util.BCEL} contains \BCEL\
         utility routines. The intention is that
         this class should grow in the future.
   \item The package \url{sandmark.watermark.CT.embed}
         contains several classes that manipulate 
         Java classfiles using \BCEL.
\end{itemize}

\subsection{Getting started}
To open a class file for editing you do the following:
\begin{listing}{1}
   sandmark.util.ClassFileCollection cfc = ...;

   cfc = new sandmark.util.ClassFileCollection(jarFile);
   org.apache.bcel.classfile.JavaClass jc = cfc.getClass(className);
   org.apache.bcel.generic.ClassGen cg = new org.apache.bcel.generic.ClassGen(jc);
   org.apache.bcel.generic.ConstantPoolGen cp = 
      new org.apache.bcel.generic.ConstantPoolGen(jc.getConstantPool());
\end{listing}
{\tt jarFile} is the jar-file that contains the class
file and {\tt className} is the name of the class
(possibly with a package prefix). 
\url{sandmark.util.ClassFileCollection} is the
repository used within \SM\ for loading and
saving class files.

A \url{org.apache.bcel.classfile.JavaClass}-object
represents the parsed class file. You can query this
object to get any information you need about the
class, for example its methods, fields, code, super-class, etc.

A \url{org.apache.bcel.generic.ClassGen}-object represents
a class that can be edited. You can add methods and fields,
you can modify existing methods, etc.

A \url{org.apache.bcel.generic.ConstantPoolGen}-object
represents a constant pool to which new constants can
be added.

When you are finished editing the class you must close
it and return it to the repository maintained by
\url{sandmark.util.ClassFileCollection}:
\begin{listingcont}
   org.apache.bcel.classfile.JavaClass jc1 = cg.getJavaClass();
   jc1.setConstantPool(cp.getFinalConstantPool());
   cfc.addClass(jc1);
\end{listingcont}


\subsection{Editing a Method}
\url{org.apache.bcel.classfile.Method} represents a 
method that has been read from a class file, while
\url{org.apache.bcel.generic.MethodGen} is the \BCEL\ class
that represents a method being edited:
\begin{listing}{1}
   org.apache.bcel.generic.ConstantPoolGen cp = ...;
   org.apache.bcel.generic.ClassGen cg = ...;

   org.apache.bcel.classfile.Method method = 
      cg.containsMethod(methodName, methodSignature);
   org.apache.bcel.generic.MethodGen mg = 
      new org.apache.bcel.generic.MethodGen(method, className, cp);
\end{listing}
Note that to open a method with {\tt containsMethod}
you need to know the class in which it is declared, 
as well as its name as well as its signature.
The reason is that a Java class can have several methods
with the same name (they're overloaded).


Alternatively, \SM\ also provides a class 
\url{sandmark.util.EditedClass.open} than 
can be used to maintain a cache of classes
and methods that have been opened by \BCEL:
\begin{listing}{1}
    sandmark.util.ClassFileCollection cfc = ...;
    sandmark.util.EditedClass ec = sandmark.util.EditedClass.open(cfc, className);
   java.util.Iterator methods = ec.methods();
   while (methods.hasNext()){
       sandmark.util.MethodID m = (sandmark.util.MethodID) methods.next();
       org.apache.bcel.generic.MethodGen mg = ec.openMethod(m);
       ....
   }
\end{listing}
A \url{sandmark.util.MethodID}-object represents a 
method within a Java application. It is essentially a tuple 
   $$\langle {\tt name}, {\tt signature}, {\tt className}\rangle.$$



\subsection{Editing Instructions}
Once you have opened a method for editing you can
get its list of instructions:
\begin{listing}{1}
   org.apache.bcel.generic.MethodGen mg = ...;

   org.apache.bcel.generic.InstructionList il = mg.getInstructionList();
   org.apache.bcel.generic.InstructionHandle[] ihs = il.getInstructionHandles();
   for(int i=0; i < ihs.length; i++) {
      org.apache.bcel.generic.InstructionHandle ih = ihs[i];
      org.apache.bcel.generic.Instruction instr = ih.getInstruction();
         ....
   }
\end{listing}
\url{org.apache.bcel.generic.InstructionList}s can be manipulated 
by appending or inserting new instruction, deleting instructions, 
etc.

All bytecode instructions in \BCEL\ are represented by their
own class. \url{org.apache.bcel.generic.IADD} is the class
that represents integer addition, for example. This allows
us to use {\tt instanceof} to classify instructions:
\begin{listing}{1}
   org.apache.bcel.generic.Instruction instr = ...;
   org.apache.bcel.generic.InstructionHandle ih = ...;

   if (instr instanceof org.apache.bcel.generic.INVOKESTATIC) {
      org.apache.bcel.generic.INVOKESTATIC call =
        (org.apache.bcel.generic.INVOKESTATIC) instr;
      String className  = call.getClassName(ec.cp);
      String methodName = call.getMethodName(ec.cp);
      String methodSig  = call.getSignature(ec.cp);

      ih.setInstruction(new org.apache.bcel.generic.NOP());
   }
\end{listing}
\url{setInstruction} is used to replace an original 
instruction with a new one.

\subsection{Local variables}
Local variables are quite complicated in Java bytecode.
A Java method can have a maximum of 256 local variables.
Note, however, that
\begin{itemize}
   \item In a virtual (non-static) method local variable
         zero (``slot \#0'') is {\tt this}.
   \item Formal arguments take up the first slots.
         In other words, in a virtual method with
         2 formal parameters, slot \#0 is {\tt this},
         slot \#1 is the first formal, 
         slot \#2 is the second formal, and
         any local variables are slots \#3,\#4,\ldots,\#255.
   \item Local variables can be reused within a method.
         For example, consider the following method:
\begin{verbatim}
   void P() {
      int x= 5;
      System.out.println(x);
      float y=5.6;
      System.out.println(y);
   }
\end{verbatim}
   A clever compiler will allocate both {\tt x} and {\tt y}
   to the same slot, since they have non-overlapping lifetimes.
\end{itemize}

For this reason, \BCEL\ requires that we indicate where
a new local variable is accessible:
\begin{listing}{1}
   org.apache.bcel.generic.Type T = ...;

   org.apache.bcel.generic.LocalVariableGen lg = 
      mg.addLocalVariable(localName, T, null, null);
   int localIndex = lg.getIndex();

   // push something here
   org.apache.bcel.generic.Instruction store = 
      new org.apache.bcel.generic.ASTORE(localIndex);

   org.apache.bcel.generic.InstructionHandle start = ...;
   lg.setStart(start);
\end{listing}
\url{org.apache.bcel.generic.LocalVariableGen} creates
a new local variable. The two {\tt null} arguments are
the locations within the method where we want the
variable to be visible. We start these locations 
out as {\tt null} (unknown) and fill them in using
the \url{setStart} method. It takes an
\url{org.apache.bcel.generic.InstructionHandle} as
input, namely the first instruction where the 
variable should be visible.


\subsection{Example 1: Creating a New Class}
Here's a longer example that shows how you create
a class from scratch.

We start by creating a new \url{org.apache.bcel.generic.ClassGen}
object representing the class. We provide the name of the class,
the name of the parent class, the interfaces it implements,
and its access flags:
\begin{listing}{1}
   int access_flags = org.apache.bcel.Constants.ACC_PUBLIC;
   String class_name = "MyClass";
   String file_name = "MyClass.java";
   String super_class_name = "java.lang.Object";
   String[] interfaces = {};

   org.apache.bcel.generic.ClassGen cg = 
      new org.apache.bcel.generic.ClassGen(
         class_name, super_class_name, file_name,
         access_flags, interfaces);
   org.apache.bcel.generic.ConstantPoolGen cp = cg.getConstantPool();
\end{listing}

We can then add a field {\tt field1} of type {\tt int}
to the new class:
\begin{listingcont}
   org.apache.bcel.generic.Type field_type = 
      org.apache.bcel.generic.Type.INT;
   String field_name = "field1"; 

   org.apache.bcel.generic.FieldGen fg =
      new org.apache.bcel.generic.FieldGen(
         access_flags, field_type, field_name, cp);
   cg.addField(fg.getField());
\end{listingcont}

Next, we create a method {\tt method1}:
\begin{listingcont}
   int method_access_flags = 
      org.apache.bcel.Constants.ACC_PUBLIC | 
      org.apache.bcel.Constants.ACC_STATIC;

   org.apache.bcel.generic.Type return_type = 
      org.apache.bcel.generic.Type.VOID;

   org.apache.bcel.generic.Type[] arg_types = 
      org.apache.bcel.generic.Type.NO_ARGS;
   String[] arg_names = {};

   String method_name = "method1";
   String class_name = "MyClass";

   org.apache.bcel.generic.InstructionList il = 
      new org.apache.bcel.generic.InstructionList();
   il.append(org.apache.bcel.generic.InstructionConstants.RETURN);

   org.apache.bcel.generic.MethodGen mg = 
      new org.apache.bcel.generic.MethodGen(
         method_access_flags, return_type,  arg_types,
         arg_names, method_name, class_name, il, cp);

   mg.setMaxStack();
   cg.addMethod(mg.getMethod());
\end{listingcont}
The method has no arguments, returns {\tt void}, and contains
only one instruction, a {\tt return}.

Note the call to {\tt mg.setMaxStack()}. With every method in a
class file is stored the maximum stack-size is needed to execute
the method. For example, a method whose body consists of the
instructions $\langle${\tt iconst\_1}, {\tt iconst\_2}, {\tt iadd},
{\tt pop}$\rangle$ (i.e. compute $1+2$ and throw away the result)
will have max-stack set to two. We can either compute this ourselves
or let \BCEL's {\tt mg.setMaxStack()} do it for us.

\subsection{Branches}
As is always the case with branches, we need to be able
to handle forward jumps. The typical way of accomplishing
this is to create a branch with unspecified target:
\begin{listing}{1}
   org.apache.bcel.generic.InstructionList il = ...;
 
   org.apache.bcel.generic.IFNULL branch = 
      new org.apache.bcel.generic.IFNULL(null);
   il.append(branch);
   ...
\end{listing}
When we have gotten to the location we want to
jump to we add a (bogus) {\tt NOP} instruction
which will serve as our branch target.
The \url{append} instruction returns a handle
to the {\tt NOP} and we use this handle to
set the target of the branch instruction:
\begin{listingcont}
   org.apache.bcel.generic.InstructionHandle h = 
      il.append(new org.apache.bcel.generic.NOP());
   branch.setTarget(h);
\end{listingcont}

\subsection{Exceptions}
Here is the generic code for building a {\tt try-catch}-block:
\begin{listing}{1}
   org.apache.bcel.generic.InstructionList il = ...;
   org.apache.bcel.generic.MethodGen mg = ...;
   String exception = "java.lang.Exception";

   org.apache.bcel.generic.InstructionHandle start_pc =
      il.append(new org.apache.bcel.generic.NOP());

   // The code that builds the try-body goes here.

   // Add code to jump out of the try-block when
   // no exception was thrown.
   org.apache.bcel.generic.GOTO branch = 
      new org.apache.bcel.generic.GOTO(null);
   org.apache.bcel.generic.InstructionHandle end_pc =
        il.append(branch);

   // Pop the exception off the stack when entering the catch block.
   org.apache.bcel.generic.InstructionHandle handler_pc =
       il.append(new org.apache.bcel.generic.POP());

   // The code that builds the catch-body goes here.

   // Add a NOP after the exception handler. This is 
   // where we will jump after we're through with the
   // try-block.
   org.apache.bcel.generic.InstructionHandle next_pc =
       il.append(new org.apache.bcel.generic.NOP());
   branch.setTarget(next_pc);

   org.apache.bcel.generic.ObjectType catch_type =
      new org.apache.bcel.generic.ObjectType(exception);

   org.apache.bcel.generic.CodeExceptionGen eg =
      mg.addExceptionHandler(start_pc, end_pc, handler_pc, catch_type);
\end{listing}
{\tt start\_pc} and {\tt end\_pc} hold the beginning and the
end of the {\tt try}-body, respectively. {\tt handler\_pc}
holds the address of the beginning of the {\tt catch}-block.

\subsection{Types}
Types (such as method signatures) are defined by the
class \url{org.apache.bcel.generic.Type}:
\begin{listing}{1}
   org.apache.bcel.generic.Type return_type = 
      org.apache.bcel.generic.Type.VOID;
   org.apache.bcel.generic.Type[] arg_types   = 
      new org.apache.bcel.generic.Type[] { 
         new org.apache.bcel.generic.ArrayType(
            org.apache.bcel.generic.Type.STRING, 1)
      };
\end{listing}

A source-level Java type is encoded into a classfile
type descriptor using the following definitions:
\begin{center}
\begin{tabular}{|l|l|l|}\hline
BaseType & Type & Interpretation \\\hline\hline
\bf B & byte & signed 8-bit integer \\\hline
\bf C & char & Unicode character\\\hline
\bf D & double & 64-bit floating point value \\\hline
\bf F & float & 32-bit floating point value\\\hline
\bf I & int &  32-bit integer \\\hline
\bf J & long & 64-bit integer\\\hline
\bf L{\tt <}{\em classname}{\tt >}; & reference & an instance of class {\tt <}{\em classname}{\tt >} \\\hline
\bf S & short & signed 16-bit integer\\\hline
\bf Z& boolean & {\tt true} pr {\tt false}\\\hline
\bf [ & reference & one array dimension\\\hline
\bf V &  & void\\\hline
\bf ({\em BaseType}*){\em BaseType} &  & method descriptor\\\hline
\end{tabular}
\end{center}

\BCEL\ has quite a number of methods that convert back
and forth between Java source format (such as {\tt "java.lang.Object[]"}),
bytecode format (such as {\tt "[Ljava/lang/Object;"}), and 
\BCEL's internal format (\url{org.apache.bcel.generic.Type}).
Unfortunately, these are by and large poorly documented.
We will describe some of the more common methods here.

\url{org.apache.bcel.generic.Type.getType} converts
from Java bytecode format to Java source format.
In other words, the following code
\begin{listing}{1}
    String S = "[Ljava/lang/Object;";
    org.apache.bcel.generic.Type T =
       org.apache.bcel.generic.Type.getType(S);
    System.out.println(T);
\end{listing}
prints out \CC{java.lang.Object[]}.

The method \url{org.apache.bcel.classfile.Utility.getSignature}
converts a type in Java bytecode format to Java source format.
The code
\begin{listing}{1}
    String type = "java.lang.Object[]";
    String S = org.apache.bcel.classfile.Utility.getSignature(type);
    System.out.println(S);
\end{listing}
prints \CC{[Ljava/lang/Object;}.

The methods \url{org.apache.bcel.generic.Type.getArgumentTypes}
and \url{org.apache.bcel.generic.Type.getReturnType} take
a type in Java bytecode format as argument and extract the
array of argument types and the return type, respectively:
\begin{listing}{1}
    String S = "(Ljava/lang/String;I)V;";
    org.apache.bcel.generic.Type[] arg_types =
       org.apache.bcel.generic.Type.getArgumentTypes(S);
    org.apache.bcel.generic.Type return_type =
       org.apache.bcel.generic.Type.getReturnType(S);
    String M = org.apache.bcel.generic.Type.getMethodSignature(return_type, arg_types);
    System.out.println(M);
\end{listing}
\url{org.apache.bcel.generic.Type.getMethodSignature} converts
the return and argument types back to a Java bytecode type string.

The method \url{org.apache.bcel.generic.Type.getSignature}
converts a \BCEL\ type to the equivalent Java bytecode 
signature. This code
\begin{listing}{1}
    org.apache.bcel.generic.Type T =
       org.apache.bcel.generic.Type.STRINGBUFFER;
    String M = T.getSignature();
    System.out.println(M);
\end{listing}
will print out \CC{Ljava/lang/StringBuffer;}.

As a convenience, \SM\ provides the method
\url{sandmark.util.javagen.Java.typeToByteCode},
that converts a {\tt type} in Java source format
to a \BCEL\ \url{org.apache.bcel.generic.Type}:
\begin{listing}{1}
   String S = org.apache.bcel.classfile.Utility.getSignature(type);
   return org.apache.bcel.generic.Type.getType(S);
\end{listing}


\subsection{Method Calls}
To make a static method call we push the arguments 
on the stack, create a constant pool reference to
the method, and then generate an \url{INVOKESTATIC}
opcode that performs the actual call:
\begin{listing}{1}
   String className  = ...;
   String methodName = ...;
   String signature  = ...;

   org.apache.bcel.generic.InstructionList il = ...;
   org.apache.bcel.generic.ConstantPoolGen cp = ...;

   // Generate code that pushes the actual arguments of the call.

   int index = cp.addMethodref(className,methodName,signature);
   org.apache.bcel.generic.INVOKESTATIC call = 
       new org.apache.bcel.generic.INVOKESTATIC(index);
   il.append(call);
\end{listing}

Making a virtual call is similar, except that we
need an object to make the call through. This
object reference is pushed on the stack prior
to the arguments:
\begin{listing}{1}
   String className  = ...;
   String methodName = ...;
   String signature  = ...;

   org.apache.bcel.generic.InstructionList il = ...;
   org.apache.bcel.generic.ConstantPoolGen cp = ...;

   // Generate code that pushes the object on the stack.

   // Generate code that pushes the actual arguments of the call.

   int index = cp.addMethodref(className,methodName,signature);
   org.apache.bcel.generic.INVOKEVIRTUAL s = 
       new org.apache.bcel.generic.INVOKEVIRTUAL(index);
   il.append(call);
\end{listing}


\subsection{Example 2: Modifying an Existing Program}
Consider the following program:
\begin{listing}{1}
   public class C {
      public static void main(String args[]) {
         int a = 5;
         int b = 6;
         int c = a + b;
         System.out.println(c);
      }
   }
\end{listing}


\subsection{Control Flow Graphs}
The {\tt JustIce} verifier that is built on top of
\BCEL\ has functions that construct control flow
graphs for methods. Unfortunately, they don't 
construct real basic blocks: each instruction is
in a block by itself.

\subsection{Miscellaneous}
\url{sandmark.util.javagen.Java.accessFlagsToByteCode}
takes a list of string modifiers
as argument and returns the corresponding access flag:
\begin{listing}{1}
   String[] attributes = {"public","static"};
   int access_flags = 
      sandmark.util.javagen.Java.accessFlagsToByteCode(attributes);
\end{listing}
